\documentclass[11pt,openany]{report}
\usepackage[margin=1in]{geometry}
\usepackage{amssymb}
\usepackage{graphicx}
\usepackage{enumerate}
\usepackage{fancyhdr}
\usepackage{array}
\usepackage{hyperref}
\usepackage{multirow}
\hypersetup{
    colorlinks,
    citecolor=black,
    filecolor=black,
    linkcolor=black,
    urlcolor=black
}
\usepackage{tabularx}
\setlength{\headheight}{15.2pt}
\pagestyle{fancy}
\title{OSOROM Peripheral Reference}
\author{The Moroso Project}
\setlength{\tabcolsep}{4pt}
\setcounter{tocdepth}{1}

\begin{document}
\lhead{The Moroso Project}
\chead{OSOROM Peripheral Reference}
\rhead{v0.1}
\maketitle
\tableofcontents

%%%%%%%%%%%%%%%%%%%%%%%%%%%%%%%%%%%%%%%%%%%%%%%%%%%%%%%%%%%%%%%%%%%%%%%%%%%%%%%%
\chapter{Onboard Peripherals}

\section{Introduction}
The OSOROM will include several onboard peripherals:
\begin{itemize}
\item A programmable timer
\item A serial port
\item A video controller
\item An SD card controller
\item A USB controller?
\end{itemize}

Access to all peripherals is through memory mapped registers; each
peripheral has its own page of physical address space for control
registers.

\section{Peripheral Reference}
\subsection{Register Map}

\begin{center}
  \begin{tabular}{|l|l|l|}
    \hline
    \multirow{3}{*}{Timer} & \verb|TIMER_COUNT| & \texttt{0x80000000} \\
                           & \verb|TIMER_TOP| & \texttt{0x80000004} \\
                           & \verb|TIMER_CONTROL| & \texttt{0x80000008} \\
    \hline
    \multirow{3}{*}{Serial Port} & \verb|SERIAL_BAUD| & \texttt{0x80001000} \\
                           & \verb|SERIAL_DATA| & \texttt{0x80001004} \\
                           & \verb|SERIAL_CONTROL| & \texttt{0x80001008} \\
    \hline
    Video Controller & & \texttt{0x80002000} \\
    \hline
    SD Controller & & \texttt{0x80003000} \\
    \hline
    USB Controller & & \texttt{0x80004000} \\
    \hline
  \end{tabular}
\end{center}
\section{Timer}
Each clock tick, the \verb|TIMER_COUNT| register is incremented. If
this value matches the value in the \verb|TIMER_TOP| register,
\verb|TIMER_COUNT| is cleared and the \verb|TIMER_INT| bit in
\verb|TIMER_CONTROL| is set.

\subsection{Registers}
\subsubsection{TIMER\_COUNT}

\begin{center}
  \begin{tabular}{|c|c|c|c|c|c|c|c|c|c|c|c|c|c|c|c|}
    \hline
    31 & 30 & 29 & 28 & 27 & 26 & 25 & 24 & 23 & 22 & 21 & 20 & 19 & 18 & 17 & 16 \\
    \hline
    \multicolumn{16}{|c|}{\texttt{TIMER\_COUNT}}\\
    \hline
    rw & rw & rw & rw & rw & rw & rw & rw & rw & rw & rw & rw & rw & rw & rw & rw\\
    \hline
    \multicolumn{16}{c}{}\\
    \hline
    15 & 14 & 13 & 12 & 11 & 10 & 9 & 8 & 7 & 6 & 5 & 4 & 3 & 2 & 1 & 0 \\
    \hline
    \multicolumn{16}{|c|}{\texttt{TIMER\_COUNT}}\\
    \hline
    rw & rw & rw & rw & rw & rw & rw & rw & rw & rw & rw & rw & rw & rw & rw & rw\\
    \hline
  \end{tabular}
\end{center}

\begin{enumerate}
\item[Bits 31:0] \verb|TIMER_COUNT|: The current count, incremented
  each tick and reset to 0 when it matches the value in
  \verb|TIMER_TOP|.
\end{enumerate}

\subsubsection{TIMER\_TOP}
\begin{center}
  \begin{tabular}{|c|c|c|c|c|c|c|c|c|c|c|c|c|c|c|c|}
    \hline
    31 & 30 & 29 & 28 & 27 & 26 & 25 & 24 & 23 & 22 & 21 & 20 & 19 & 18 & 17 & 16 \\
    \hline
    \multicolumn{16}{|c|}{\texttt{TIMER\_TOP}}\\
    \hline
    rw & rw & rw & rw & rw & rw & rw & rw & rw & rw & rw & rw & rw & rw & rw & rw\\
    \hline
    \multicolumn{16}{c}{}\\
    \hline
    15 & 14 & 13 & 12 & 11 & 10 & 9 & 8 & 7 & 6 & 5 & 4 & 3 & 2 & 1 & 0 \\
    \hline
    \multicolumn{16}{|c|}{\texttt{TIMER\_TOP}}\\
    \hline
    rw & rw & rw & rw & rw & rw & rw & rw & rw & rw & rw & rw & rw & rw & rw & rw\\
    \hline
  \end{tabular}
\end{center}

\begin{enumerate}
  \item[Bits 31:0] \verb|TIMER_TOP|: The top value for the timer. When \verb|TIMER_COUNT|
              is equal to this value, \verb|TIMER_COUNT| will be reset to 0 and
              the \verb|TIMER_INT| bit of \verb|TIMER_CONTROL| will be set.
\end{enumerate}

\subsubsection{TIMER\_CONTROL}
\begin{center}
  \begin{tabular}{|c|c|c|c|c|c|c|c|c|c|c|c|c|c|c|c|}
    \hline
    31 & 30 & 29 & 28 & 27 & 26 & 25 & 24 & 23 & 22 & 21 & 20 & 19 & 18 & 17 & 16 \\
    \hline
    \multicolumn{16}{|c|}{\multirow{2}{*}{(reserved)}}\\
    \multicolumn{16}{|c|}{}\\
    \hline
    \multicolumn{16}{c}{}\\
    \hline
    15 & 14 & 13 & 12 & 11 & 10 & 9 & 8 & 7 & 6 & 5 & 4 & 3 & 2 & 1 & 0 \\
    \hline
    \multicolumn{13}{|c|}{\multirow{2}{*}{(reserved)}} & \texttt{TIMER\_EN} & \texttt{TIMER\_INT\_EN} & \texttt{TIMER\_INT} \\
    \cline{14-16}
    \multicolumn{13}{|c|}{} & rw & rw & rw\\
    \hline
  \end{tabular}
\end{center}

\begin{enumerate}
  \item[Bit 2] \verb|TIMER_EN|: Timer enable bit. The counter will be paused if
    this bit is set to 0.
  \item[Bit 1] \verb|TIMER_INT_EN|: Timer interrupt enable. If this bit and
    \verb|TIMER_INT| are both set, a timer interrupt will be generated.
  \item[Bit 0] \verb|TIMER_INT|: Timer interrupt flag. Set by hardware
    when \verb|TIMER_COUNT| is equal to \verb|TIMER_TOP|.
\end{enumerate}

\section{Serial Port}

The serial port contains two internal 8-bit registers, the receive and
transmit buffer registers. These cannot be accessed directly, but instead
are accessed through the \verb|DATA| register.

When a value is written to the \verb|DATA| register, it is written
through to the transmit buffer. If a transmission is not in progress,
a transmission will begin with the written value, and the \verb|TX_EMPTY|
bit will be set to indicate that the transmit buffer is ready for
another write. If a value is written to \verb|DATA| and a transmission
is in progress, the value will be written to the transmit buffer and
the \verb|TX_EMPTY| bit will be cleared. When the transmission
ends, if \verb|TX_EMPTY| is clear, a new transmission will begin
with the value in the transmit buffer, and the \verb|TX_EMPTY|
and \verb|TX_COMPLETE| bits will be set.

Reads from the \verb|DATA| register read the value in the receive buffer.
When a byte is received, the value is stored in this buffer and the
\verb|RX_COMPLETE| bit is set.

\subsubsection{SERIAL\_BAUD}

\begin{center}
  \begin{tabular}{|c|c|c|c|c|c|c|c|c|c|c|c|c|c|c|c|}
    \hline
    31 & 30 & 29 & 28 & 27 & 26 & 25 & 24 & 23 & 22 & 21 & 20 & 19 & 18 & 17 & 16 \\
    \hline
    \multicolumn{16}{|c|}{\texttt{SERIAL\_BAUD}}\\
    \hline
    rw & rw & rw & rw & rw & rw & rw & rw & rw & rw & rw & rw & rw & rw & rw & rw\\
    \hline
    \multicolumn{16}{c}{}\\
    \hline
    15 & 14 & 13 & 12 & 11 & 10 & 9 & 8 & 7 & 6 & 5 & 4 & 3 & 2 & 1 & 0 \\
    \hline
    \multicolumn{16}{|c|}{\texttt{SERIAL\_BAUD}}\\
    \hline
    rw & rw & rw & rw & rw & rw & rw & rw & rw & rw & rw & rw & rw & rw & rw & rw\\
    \hline
  \end{tabular}
\end{center}

\begin{enumerate}
\item[Bits 31:0] \verb|SERIAL_BAUD|: A value that will determine the baud
rate according to some formula we'll figure out later.
\end{enumerate}

\subsubsection{SERIAL\_DATA}

\begin{center}
  \begin{tabular}{|c|c|c|c|c|c|c|c|c|c|c|c|c|c|c|c|}
    \hline
    31 & 30 & 29 & 28 & 27 & 26 & 25 & 24 & 23 & 22 & 21 & 20 & 19 & 18 & 17 & 16 \\
    \hline
    \multicolumn{16}{|c|}{\multirow{2}{*}{(reserved)}}\\
    \multicolumn{16}{|c|}{}\\
    \hline
    \multicolumn{16}{c}{}\\
    \hline
    15 & 14 & 13 & 12 & 11 & 10 & 9 & 8 & 7 & 6 & 5 & 4 & 3 & 2 & 1 & 0 \\
    \hline
    \multicolumn{7}{|c|}{\multirow{2}{*}{(reserved)}} & \texttt{P} & \multicolumn{8}{c|}{\texttt{DATA}}\\
    \cline{8-16}
    \multicolumn{7}{|c|}{} & rw & rw & rw & rw & rw & rw & rw & rw & rw\\
    \hline
  \end{tabular}
\end{center}

\begin{enumerate}
\item[Bit 8] \verb|P|: Reserved for use as a parity bit. Reads as 0.
\item[Bits 7:0] \verb|DATA|: Data register. See description above.
\end{enumerate}

\subsubsection{SERIAL\_CONTROL}
\begin{center}
  \begin{tabular}{|c|c|c|c|c|c|c|c|c|}
    \hline
    31 & $\cdots$ & 22 & 21 & 20 & 19 & 18 & 17 & 16 \\
    \hline
    \multicolumn{9}{|c|}{\multirow{2}{*}{(reserved)}}\\
    \multicolumn{9}{|c|}{}\\
    \hline
    \multicolumn{9}{c}{}\\
    \hline
    15 & $\cdots$ & 6 & 5 & 4 & 3 & 2 & 1 & 0 \\
    \hline
    \multicolumn{3}{|c|}{\multirow{2}{*}{(reserved)}} & \texttt{RXC\_INT\_EN} & \texttt{RX\_COMPLETE} & \texttt{TXE\_INT\_EN} & \texttt{TX\_EMPTY} & \texttt{TXC\_INT\_EN} & \texttt{TX\_COMPLETE} \\
    \cline{4-9}
    \multicolumn{3}{|c|}{} & rw & rw & rw & rw & rw & rw\\
    \hline
  \end{tabular}
\end{center}

\begin{enumerate}
\item[Bit 5] \verb|RXC_INT_EN|: Interrupt enable for \verb|RX_COMPLETE|.
\item[Bit 4] \verb|RX_COMPLETE|: Set by hardware when a receive is
  complete. Should be cleared by software when the value in
  \verb|DATA| is read.
\item[Bit 3] \verb|TXE_INT_EN|: Interrupt enable for \verb|TX_EMPTY|.
\item[Bit 2] \verb|TX_EMPTY|: Cleared by hardware when a value is
  written to \verb|DATA| and a transmission is in progress. Set by
  hardware when a transmission begins.
\item[Bit 1] \verb|TXC_INT_EN|: Interrupt enable for \verb|TX_COMPLETE|.
\item[Bit 0] \verb|TX_COMPLETE|: Set by hardware when a transmission
  completes.
\end{enumerate}


\section{Video Controller}

\section{SD Card Controller}

\section{USB Controller}


\end{document}
