\documentclass[11pt,openany]{report}
\usepackage[margin=1in]{geometry}
\usepackage{amssymb}
\usepackage{graphicx}
\usepackage{enumerate}
\usepackage{fancyhdr}
\usepackage{array}
\usepackage{hyperref}
\usepackage{multirow}
\hypersetup{
    colorlinks,
    citecolor=black,
    filecolor=black,
    linkcolor=black,
    urlcolor=black
}
\usepackage{tabularx}
\setlength{\headheight}{15.2pt}
\pagestyle{fancy}
\title{OSOROM Peripheral Reference}
\author{The Moroso Project}
\setlength{\tabcolsep}{4pt}
\setcounter{tocdepth}{1}

\newcommand\iic{I$^2$C}

\begin{document}
\lhead{The Moroso Project}
\chead{OSOROM Peripheral Reference}
\rhead{v0.1}
\maketitle
\tableofcontents

%%%%%%%%%%%%%%%%%%%%%%%%%%%%%%%%%%%%%%%%%%%%%%%%%%%%%%%%%%%%%%%%%%%%%%%%%%%%%%%%
\chapter{Onboard Peripherals}

\section{Introduction}
The OSOROM includes several onboard peripherals:
\begin{itemize}
\item LED/button/switch IO
\item A serial port
\item An \iic\ controller
\item An SD card controller
\item An audio controller
\item A video controller
\item An interrupt controller
\end{itemize}
and will likely eventually have:
\begin{itemize}
\item A programmable timer
\item A USB controller?
\end{itemize}

Access to all peripherals is through memory mapped registers; each
peripheral has its own page of physical address space for control
registers.

\section{Peripheral Reference}
\subsection{Peripheral Map}
Each peripheral instance has an index; that index is used to determine
the offset of its page of memory mapped registers, and is also used
in the interrupt controller to enable or disable interrupts from that
peripheral instance. The peripheral instances and their indices are:

\begin{center}
  \begin{tabular}{|l|l|l|}
    \hline
    Index & Name & Description \\
    \hline
    0 & \verb|LEDSW| & LEDs and switches \\
    \hline
    1 & \verb|UART| & Serial UART \\
    \hline
    2 & \verb|I2C| & \iic\ controller \\
    \hline
    3 & \verb|SD| & SD card controller \\
    \hline
    4 & \verb|AUDIO| & Audio controller \\
    \hline
    5 & \verb|VIDEO| & Video controller \\
    \hline
    6 & \verb|INT| & Interrupt controller \\
    \hline
    7 & \verb|USB| & USB controller (reserved) \\
    \hline
    8-15 & \verb|TIM0|-\verb|TIM7| & Timers (reserved) \\
    \hline
  \end{tabular}
\end{center}

The base address for the memory mapped registers of a peripheral
is the peripheral's index, in pages, from the start of the peripheral
region (\texttt{0x80000000}). So, for example, interrupt registers
(peripheral index 6) begin at \texttt{0x80006000}.

\chapter{Peripherals}

\section{LED/switch/button interface}
The Moroso board has sixteen LEDs (eight green and eight red), as well as four buttons
and ten switches; all are accessed through the \verb|LEDSW| peripheral. Interrupts
can be configured on state changes of any button or switch. In the future, more
control over interrupts (such as the ability to trigger on a rising/falling edge)
may be added.

\subsection{Registers}
\subsubsection{\texttt{0x0}: LEDS}
\begin{center}
  \begin{tabular}{|c|c|c|c|c|c|c|c|c|c|c|c|c|c|c|c|}
    \hline
    31 & 30 & 29 & 28 & 27 & 26 & 25 & 24 & 23 & 22 & 21 & 20 & 19 & 18 & 17 & 16 \\
    \hline
    \multicolumn{16}{|c|}{\multirow{2}{*}{(reserved)}}\\
    \multicolumn{16}{|c|}{}
    \\
    \hline
    \multicolumn{16}{c}{}\\
    \hline
    15 & 14 & 13 & 12 & 11 & 10 & 9 & 8 & 7 & 6 & 5 & 4 & 3 & 2 & 1 & 0 \\
    \hline
    \multicolumn{8}{|c|}{\texttt{LEDS\_R}} & \multicolumn{8}{c|}{\texttt{LEDS\_G}} \\
    \hline
    rw & rw & rw & rw & rw & rw & rw & rw & rw & rw & rw & rw & rw & rw & rw & rw\\
    \hline
  \end{tabular}
\end{center}

\begin{enumerate}
\item[Bits 15:8] \verb|LEDS_R|: The value of the red LEDs (MSB corresponding to the leftmost red LED).
\item[Bits 7:0] \verb|LEDS_G|: The value of the green LEDs (MSB corresponding to the leftmost green LED).
\end{enumerate}

\subsubsection{\texttt{0x4}: SW\_BUTTONS}
\begin{center}
  \begin{tabular}{|c|c|c|c|c|c|c|c|c|c|c|c|c|c|c|c|}
    \hline
    31 & 30 & 29 & 28 & 27 & 26 & 25 & 24 & 23 & 22 & 21 & 20 & 19 & 18 & 17 & 16 \\
    \hline
    \multicolumn{6}{|c|}{\multirow{2}{*}{(reserved)}} & \multicolumn{10}{c|}{\texttt{SWITCHES}}\\
    \cline{7-16}
    \multicolumn{6}{|c|}{} & rw & rw & rw & rw & rw & rw & rw & rw & rw & rw\\
    \hline
    \multicolumn{16}{c}{}\\
    \hline
    15 & 14 & 13 & 12 & 11 & 10 & 9 & 8 & 7 & 6 & 5 & 4 & 3 & 2 & 1 & 0 \\
    \hline
    \multicolumn{12}{|c|}{\multirow{2}{*}{(reserved)}} & \multicolumn{4}{c|}{\texttt{BUTTONS}} \\
    \cline{13-16}
    \multicolumn{12}{|c|}{} & rw & rw & rw & rw\\
    \hline
  \end{tabular}
\end{center}

\begin{enumerate}
\item[Bits 25:16] \verb|SWITCHES|: The value of each switch (MSB corresponding to leftmost switch).
\item[Bits 3:0] \verb|BUTTONS|: The value of each button (MSB corresponding to the leftmost button).
\end{enumerate}

\subsection{Register map}
\begin{center}
  \begin{tabular}{|l|l|}
    \hline
    \textbf{Offset} & \textbf{Register name} \\
    \hline
    \texttt{0x0} & \texttt{LEDS} \\
    \hline
    \texttt{0x4} & \texttt{SW\_BUTTONS} \\
    \hline
  \end{tabular}
\end{center}

\section{Serial Port}

The serial port contains two internal 8-bit registers, the receive and
transmit buffer registers. These cannot be accessed directly, but instead
are accessed through the \verb|DATA| register.

When a value is written to the \verb|DATA| register, it is written
through to the transmit buffer. If a transmission is not in progress,
a transmission will begin with the written value, and the \verb|TX_EMPTY|
bit will be set to indicate that the transmit buffer is ready for
another write. If a value is written to \verb|DATA| and a transmission
is in progress, the value will be written to the transmit buffer and
the \verb|TX_EMPTY| bit will be cleared. When the transmission
ends, if \verb|TX_EMPTY| is clear, a new transmission will begin
with the value in the transmit buffer, and the \verb|TX_EMPTY|
and \verb|TX_COMPLETE| bits will be set.

Reads from the \verb|DATA| register read the value in the receive buffer.
When a byte is received, the value is stored in this buffer and the
\verb|RX_COMPLETE| bit is set.

Bits in the status register can be cleared by writing a $1$ to
them.

\subsubsection{\texttt{0x0}: DATA}

\begin{center}
  \begin{tabular}{|c|c|c|c|c|c|c|c|c|c|c|c|c|c|c|c|}
    \hline
    31 & 30 & 29 & 28 & 27 & 26 & 25 & 24 & 23 & 22 & 21 & 20 & 19 & 18 & 17 & 16 \\
    \hline
    \multicolumn{16}{|c|}{\multirow{2}{*}{(reserved)}}\\
    \multicolumn{16}{|c|}{}\\
    \hline
    \multicolumn{16}{c}{}\\
    \hline
    15 & 14 & 13 & 12 & 11 & 10 & 9 & 8 & 7 & 6 & 5 & 4 & 3 & 2 & 1 & 0 \\
    \hline
    \multicolumn{7}{|c|}{\multirow{2}{*}{(reserved)}} & \texttt{P} & \multicolumn{8}{c|}{\texttt{DATA}}\\
    \cline{8-16}
    \multicolumn{7}{|c|}{} & rw & rw & rw & rw & rw & rw & rw & rw & rw\\
    \hline
  \end{tabular}
\end{center}

\begin{enumerate}
\item[Bit 8] \verb|P|: Reserved for use as a parity bit. Reads as 0.
\item[Bits 7:0] \verb|DATA|: Data register. See description above.
\end{enumerate}

\subsubsection{\texttt{0x4}: CSR}
\begin{center}
  \begin{tabular}{|c|c|c|c|c|c|c|c|}
    \hline
    31 & $\cdots$ & 21 & 20 & 19 & 18 & 17 & 16  \\
    \hline
    \multicolumn{8}{|c|}{\multirow{2}{*}{(reserved)}}\\
    \multicolumn{8}{|c|}{}\\
    \hline
    \multicolumn{7}{c}{}\\
    \hline
    15 & $\cdots$ & 5 & 4 & 3 & 2 & 1 & 0 \\
    \hline
    \multicolumn{3}{|c|}{\multirow{2}{*}{(reserved)}} & \texttt{RX\_ERROR} & \texttt{RX\_COMPLETE} & \texttt{RX\_ENABLE} & \texttt{TX\_EMPTY} & \texttt{TX\_COMPLETE} \\
    \cline{4-8}
    \multicolumn{3}{|c|}{} & rw & rw & rw & r & rw\\
    \hline
  \end{tabular}
\end{center}

\begin{enumerate}
\item[Bit 4] \verb|RX_ERROR|: Set by hardware on receive error.
\item[Bit 3] \verb|RX_COMPLETE|: Set by hardware when a byte is received.
\item[Bit 2] \verb|RX_ENABLE|: Enable UART receiver.
\item[Bit 1] \verb|TX_EMPTY|: Indicates whether the TX buffer is empty. Set and cleared by hardware.
\item[Bit 0] \verb|TX_COMPLETE|: Set by hardware when a byte transmission finishes.
\end{enumerate}


\subsubsection{\texttt{0x8}: BAUD (not yet implemented)}

\begin{center}
  \begin{tabular}{|c|c|c|c|c|c|c|c|c|c|c|c|c|c|c|c|}
    \hline
    31 & 30 & 29 & 28 & 27 & 26 & 25 & 24 & 23 & 22 & 21 & 20 & 19 & 18 & 17 & 16 \\
    \hline
    \multicolumn{16}{|c|}{\texttt{SERIAL\_BAUD}}\\
    \hline
    rw & rw & rw & rw & rw & rw & rw & rw & rw & rw & rw & rw & rw & rw & rw & rw\\
    \hline
    \multicolumn{16}{c}{}\\
    \hline
    15 & 14 & 13 & 12 & 11 & 10 & 9 & 8 & 7 & 6 & 5 & 4 & 3 & 2 & 1 & 0 \\
    \hline
    \multicolumn{16}{|c|}{\texttt{SERIAL\_BAUD}}\\
    \hline
    rw & rw & rw & rw & rw & rw & rw & rw & rw & rw & rw & rw & rw & rw & rw & rw\\
    \hline
  \end{tabular}
\end{center}

\begin{enumerate}
\item[Bits 31:0] \verb|SERIAL_BAUD|: Determines the baud rate as a function
  of the CPU clock:
  \[
  \textrm{Baud rate} = \frac{f_{\texttt{CPU}}}{\texttt{SERIAL\_BAUD} + 1}
  \]
\end{enumerate}

\subsection{Register map}
\begin{center}
  \begin{tabular}{|l|l|}
    \hline
    \textbf{Offset} & \textbf{Register name} \\
    \hline
    \texttt{0x0} & \texttt{DATA} \\
    \hline
    \texttt{0x4} & \texttt{CSR} \\
    \hline
  \end{tabular}
\end{center}

\section{Timer}
Each clock tick, the \verb|COUNT| register is incremented. If
this value matches the value in the \verb|TOP| register,
\verb|COUNT| is cleared and an interrupt is signaled.

\subsection{Registers}
\subsubsection{\texttt{0x0}: COUNT}

\begin{center}
  \begin{tabular}{|c|c|c|c|c|c|c|c|c|c|c|c|c|c|c|c|}
    \hline
    31 & 30 & 29 & 28 & 27 & 26 & 25 & 24 & 23 & 22 & 21 & 20 & 19 & 18 & 17 & 16 \\
    \hline
    \multicolumn{16}{|c|}{\texttt{TIMER\_COUNT}}\\
    \hline
    rw & rw & rw & rw & rw & rw & rw & rw & rw & rw & rw & rw & rw & rw & rw & rw\\
    \hline
    \multicolumn{16}{c}{}\\
    \hline
    15 & 14 & 13 & 12 & 11 & 10 & 9 & 8 & 7 & 6 & 5 & 4 & 3 & 2 & 1 & 0 \\
    \hline
    \multicolumn{16}{|c|}{\texttt{TIMER\_COUNT}}\\
    \hline
    rw & rw & rw & rw & rw & rw & rw & rw & rw & rw & rw & rw & rw & rw & rw & rw\\
    \hline
  \end{tabular}
\end{center}

\begin{enumerate}
\item[Bits 31:0] \verb|TIMER_COUNT|: The current count, incremented
  each tick and reset to 0 when it matches the value in
  \verb|TIMER_TOP|.
\end{enumerate}

\subsubsection{\texttt{0x4}: TOP}
\begin{center}
  \begin{tabular}{|c|c|c|c|c|c|c|c|c|c|c|c|c|c|c|c|}
    \hline
    31 & 30 & 29 & 28 & 27 & 26 & 25 & 24 & 23 & 22 & 21 & 20 & 19 & 18 & 17 & 16 \\
    \hline
    \multicolumn{16}{|c|}{\texttt{TIMER\_TOP}}\\
    \hline
    rw & rw & rw & rw & rw & rw & rw & rw & rw & rw & rw & rw & rw & rw & rw & rw\\
    \hline
    \multicolumn{16}{c}{}\\
    \hline
    15 & 14 & 13 & 12 & 11 & 10 & 9 & 8 & 7 & 6 & 5 & 4 & 3 & 2 & 1 & 0 \\
    \hline
    \multicolumn{16}{|c|}{\texttt{TIMER\_TOP}}\\
    \hline
    rw & rw & rw & rw & rw & rw & rw & rw & rw & rw & rw & rw & rw & rw & rw & rw\\
    \hline
  \end{tabular}
\end{center}

\begin{enumerate}
  \item[Bits 31:0] \verb|TIMER_TOP|: The top value for the timer. When \verb|TIMER_COUNT|
              is equal to this value, \verb|TIMER_COUNT| will be reset to 0 and
              an interrupt will be signaled.
\end{enumerate}

\subsubsection{\texttt{0x8}: CONTROL}
\begin{center}
  \begin{tabular}{|c|c|c|c|c|c|c|c|c|c|c|c|c|c|c|c|}
    \hline
    31 & 30 & 29 & 28 & 27 & 26 & 25 & 24 & 23 & 22 & 21 & 20 & 19 & 18 & 17 & 16 \\
    \hline
    \multicolumn{16}{|c|}{\multirow{2}{*}{(reserved)}}\\
    \multicolumn{16}{|c|}{}\\
    \hline
    \multicolumn{16}{c}{}\\
    \hline
    15 & 14 & 13 & 12 & 11 & 10 & 9 & 8 & 7 & 6 & 5 & 4 & 3 & 2 & 1 & 0 \\
    \hline
    \multicolumn{15}{|c|}{\multirow{2}{*}{(reserved)}} & \texttt{TIMER\_EN} \\
    \cline{16-16}
    \multicolumn{15}{|c|}{} & rw \\
    \hline
  \end{tabular}
\end{center}

\begin{enumerate}
  \item[Bit 0] \verb|TIMER_EN|: Timer enable bit. The counter will be paused if
    this bit is set to 0.
\end{enumerate}

\subsection{Register map}
\begin{center}
  \begin{tabular}{|l|l|}
    \hline
    \textbf{Offset} & \textbf{Register name} \\
    \hline
    \texttt{0x0} & \texttt{COUNT} \\
    \hline
    \texttt{0x4} & \texttt{TOP} \\
    \hline
    \texttt{0x8} & \texttt{CONTROL} \\
    \hline
  \end{tabular}
\end{center}

\section{Video Controller}

\section{SD Card Controller}

\section{USB Controller}


\end{document}
